\documentclass[xcolor=dvipsnames, 11pt]{beamer}
\usetheme{AnnArbor}

\usecolortheme{crane}
\setbeamertemplate{caption}[numbered]
\usepackage{graphicx}
\graphicspath{{./images/}}
\usepackage[utf8]{inputenc}
\usepackage{amsmath}
%\usepackage{tabto}
\usepackage{amsfonts}
\usepackage{amssymb}
\usepackage{amsthm}
\usepackage{xcolor}
\usepackage{tikz}
\usepackage{tikz-cd}
\usetikzlibrary{arrows}
\usepackage[all,knot,arc,poly]{xy}
\usetikzlibrary{decorations.markings,arrows,chains,matrix,positioning,scopes}
\let\amsamp=&

% \include{defs}

\include{biblio}

\DeclareMathOperator{\Ima}{Im}

\title{Topological Data Analysis}
\subtitle{Introducing Persistent Homology}
\author{Kejsi Jonuzaj}
\institute{American University in Bulgaria}
\date{\today}

\begin{document}
 
 \begin{frame}
 \maketitle
 \end{frame}
 
 \begin{frame}
 
  \frametitle{Outline}
  \tableofcontents
  
  \end{frame}
  
  \section{Simplicial Homology}
  \subsection{$n-simplex$}
  
  \begin{frame}
  \frametitle{Standard Simplex - \emph{n-simplex}}
  A $n-simplex$ is denoted by $[v_0, v_1, ... , v_n]$
        \begin{figure}
         \centering

            

		      \begin{tikzpicture}[line join = round, line cap = round]

                        % 0-simplex
                        \coordinate [label=below:$v_0$] (0) at (0,0);
                        % 1-simplex
                        \coordinate [label=below:$v_0$] (1) at (1,0);
                        \coordinate [label=below:$v_1$] (2) at (3,0);
                        % 2-simplex
                        \coordinate [label=below:$v_0$] (3) at (4,0);
                        \coordinate [label=below:$v_1$] (4) at (6,0);
                        \coordinate [label=above:$v_2$] (5) at (5,{sqrt(3)});
                        % 3-simpex
                        \coordinate [label=above:$v_3$] (6) at (8,{sqrt(2)},0);
                        \coordinate [label=left:$v_0$] (7) at ({-.5*sqrt(3)+8},0,-.5);
                        \coordinate [label=below:$v_1$] (8) at (8,0,1);
                        \coordinate [label=right:$v_2$] (9) at ({.5*sqrt(3)+8},0,-.5);

                        \begin{scope}[decoration={markings,mark=at position 0.5 with
                            {\arrow{to}}}]
                            % 0-simplex
                            \draw[fill] (0) circle [radius=0.03];
                            % 1-simplex
                            \draw[fill] (1) circle [radius=0.025];
                            \draw[fill] (2) circle [radius=0.025];
                            \draw[thick, postaction={decorate}] (1)--(2);
                            % 2-simplex
                            \draw[fill] (3) circle [radius=0.025];
                            \draw[fill] (4) circle [radius=0.025];
                            \draw[fill] (5) circle [radius=0.025];
                            \draw[thick, postaction={decorate}] (3)--(4);
                            \draw[thick, postaction={decorate}] (3)--(5);
                            \draw[thick, postaction={decorate}] (4)--(5);
                            \filldraw[opacity=.2, gray] (3) --  (4) --  (5) -- cycle;
                            % 3-simplex
                            \draw[fill] (6) circle [radius=0.03];
                            \draw[fill] (7) circle [radius=0.025];
                            \draw[fill] (8) circle [radius=0.025];
                            \draw[fill] (9) circle [radius=0.025];
                            \draw[thick, densely dotted,postaction={decorate}] (7)--(9);
                            \draw[thick, postaction={decorate}] (7)--(8);
                            \draw[thick, postaction={decorate}] (7)--(6);
                            \draw[thick, postaction={decorate}] (8)--(9);
                            \draw[thick, postaction={decorate}] (8)--(6);
                            \draw[thick, postaction={decorate}] (9)--(6);
                        \end{scope}

                    \end{tikzpicture}
                \caption{0-simplex, 1-simpex, 2-simplex, 3-simplex}
        \end{figure}
        
    
		     
    \end{frame}
 
  \subsection{$\Delta-complex$}
  \begin{frame}
  \frametitle{$\Delta-complex$}
%   \begin{definition}[$\Delta$-complex]
% 		      	A $\Delta-complex$ structure on a space X is a collection of maps $\sigma_\alpha: \Delta^n \rightarrow X $ , with n depending on the index $\alpha$, such that:
%                     \begin{enumerate}
%                         \item The restriction $\sigma_\alpha | \mathring{\Delta^n}$ is                      injective, and each point of X is in the image of exactly one such restriction $\sigma_\alpha | \mathring{\Delta^n}$.
%                         \item Each restriction of $\sigma_\alpha$ to a face of $\Delta^n$ is one of the  maps
%                         $\sigma_\beta: \Delta^{n-1} \rightarrow X $. Here we are identifying the face of $\Delta^n$ with $\Delta^{n-1}$ by the canonical linear homeomorphism between them that preserves the ordering of the vertices.
%                         \item A set $A \subset X$ is open iff $\sigma^{-1}_{\alpha}(A)$ is open in $\Delta^n$ for each $\sigma_\alpha$
%                     \end{enumerate}
% 
% 		      \end{definition}
\begin{figure}
    \includegraphics[width=50mm,scale=0.5]{Simplicial_complex_example.png}
    \caption{A simplicial 3-complex}
    \label{fig:fig1}
    \end{figure}
		       
 \end{frame}
  \subsection{Chain Complex}
  \begin{frame}
   \begin{definition}[Chain complex]
		      	   Complex of abelian groups.\\
		      	   A chain complex is a sequence of homomorphisms of abelian groups:
		      	   \[
                        \xymatrix{
                            {...}  \ar[r] & 
                            C_{n+1}  \ar[r]^{\partial_{n+1}} & 
                            C_n  \ar[r]^{\partial_n} & 
                            C_{n-1}  \ar[r]^{\partial_{n-1}} & 
                            {...}  \ar[r] & 
                            C_1  \ar[r]^{\partial_1} & 
                            C_0  \ar[r]^{\partial_0 = 0}
                            & 0 \\ }
                   \]
		      	   where \(\partial_n\partial_{n+1}=0\) for each n  in $\mathbb{Z}$. The equation
		      	   \(\partial_n\partial_{n+1}=0\) is equivalent to the inclusion $ \Ima\partial_{n+1} \subset \ker\partial_n $.
		      \end{definition}
            where the boundary homomorphisms $ \partial_n $ of $[v_0, v_1, ... , v_n]$ is a defined as
            $ \Sigma_i (-1)^i [v_0, ... , \hat{v_i}, ... , v_n] $ where the '\^{}' symbol denotes the absence 
            of that vertex.
 \end{frame}
 
 \begin{frame}
 \frametitle{Boundary operator}
 \begin{columns}
        \begin{column}{0.4\textwidth}
            \begin{center}

                \begin{tikzpicture}[line join = round, line cap = round]
              
                                % 2-simplex
                                \coordinate [label=below:$v_0$] (3) at (4,0);
                                \coordinate [label=below:$v_1$] (4) at (6,0);
                                \coordinate [label=above:$v_2$] (5) at (5,{sqrt(3)});
                                \coordinate (0) at (4.8, 0.5);
                               
                                
                                \begin{scope}[decoration={markings,mark=at position 0.5 with {\arrow{to}}}]
                                    
                                  
                                    \draw[fill] (3) circle [radius=0.025];
                                    \draw[fill] (4) circle [radius=0.025];
                                    \draw[fill] (5) circle [radius=0.025];
                                    \draw[thick, postaction={decorate}] (3)--(4) node [midway, below] {a};
                                    \draw[thick, postaction={decorate}] (3)--(5) node [near start, above=10pt] {c};

                                    \draw[thick, postaction={decorate}] (4)--(5) node [near start, above=10pt] {b};
                                    \draw[->,>=latex'] (0) arc[radius=0.25,start angle=225,delta angle=300];
                                    
                                \end{scope}
                \end{tikzpicture} 
            \end{center}
        \end{column}
\begin{column}{0.6\textwidth}  %%<--- here
     $\partial[v_0, v_1] = v_1 - v_0$ \\~\\
     $\partial[v_0, v_1, v_2] = [v_0, v_1] + [v_1, v_2] - [v_0, v_2]$ \\ 
    
\end{column}
\end{columns}
  
 \end{frame}

 \begin{frame}
  \frametitle{Homology of a Chain Complex}
        \begin{definition}[Homology Group]
            The n-th homology group of the chain complex is defined as the quotient group
            \[ 
            H_n = \frac{Z_n}{B_n} = \frac{\ker\partial_n}{\Ima\partial_{n+1}}
            \]
            Elements of $Z_n$ are called cycles and elements of $B_n$ are called boundaries.
  \end{definition}
      Elements of $H_n$ are cosets of $\Ima\partial_{n+1}$,
      called homology classes. Two cycles representing the same homology class are said
      to be homologous. This means their difference is a boundary.
 \end{frame}

  \subsection{Computing Homology}

 \begin{frame}
 \frametitle{Computing Homology of $S^1$ in $\mathbb{Z}$}
 
Space $\mathcal{X} = S^1$  \\~\\
        \begin{figure}
            \begin{tikzpicture}[line join = round, line cap = round]
              
                                % 2-simplex
                                \coordinate [label=below:$v_0$] (3) at (3,0);
                                \coordinate [label=below:$v_1$] (4) at (5,0);
                                \coordinate [label=above:$v_2$] (5) at (4,{sqrt(3)});
                                \coordinate [label=above:$\simeq$] (1) at (2,0.5);
                                \coordinate (0) at (3.8, 0.5);
                                
                                \begin{scope}[decoration={markings,mark=at position 0.5 with {\arrow{to}}}]
                                    
                                    \draw[fill] (1) circle [radius=0];
                                    \draw[fill] (3) circle [radius=0.025];
                                    \draw[fill] (4) circle [radius=0.025];
                                    \draw[fill] (5) circle [radius=0.025];
                                    
                                    \draw[thick, postaction={decorate}] (3)--(4) node [midway, below] {a};
                                    \draw[thick, postaction={decorate}] (3)--(5) node [near start, above=10pt] {c};
                                    \draw[thick, postaction={decorate}] (4)--(5) node [near start, above=10pt] {b};
                                    \draw[->,>=latex'] (0) arc[radius=0.25,start angle=225,delta angle=300];
                                    \draw[thick] (0,{sqrt(3)/2}) circle (1.2cm);

                                    
                            \end{scope}
                \end{tikzpicture} 
                \caption{Triangulation of $S^1$}
            \end{figure}
            
 \end{frame}
 
 \begin{frame}
 
   We can construct the following chain complex which is a sequence of homomorphisms of abelian groups: 

	\[
		\xymatrix{
			0  \ar[r]^{\partial_2 = 0} & 
			C_1  \ar[r]^{\partial_1} & 
			C_0  \ar[r]^{\partial_0 = 0}
			& 0 \\ }
	\]

 where \(\partial_n\partial_{n+1}=0\) for each n  in $\mathbb{Z}$ and 
 
			\[
				\left|
				  \begin{array}{l}
				  	C_0= \langle v_0, v_1, v_2 \rangle \\
				  	C_1=\langle a, b, c \rangle \\
                    C_n=\{0\} \quad \forall n \geqslant 2 
				  \end{array}
				\right., 
			\]

			\[
                \xymatrix{
                    0  \ar[r]^{\partial_2 = 0} & 
                    \mathbb{Z}^{\oplus^3}  \ar[r]^{\partial_1} & 
                    \mathbb{Z}^{\oplus^3}  \ar[r]^{\partial_0 = 0}
                    & 0 \\ }
	        \]
	        
 \end{frame}
 
    \begin{frame}
    \frametitle{$H_0$ - \# of connected components}
            The n-th homology group is defined as $H_n = \frac{\ker\partial_n}{\Ima\partial_{n+1}}$. \\


First, let's compute $H_0$: 

$\ker\partial_0 = C_0 = \langle v_0, v_1, v_2 \rangle$ since $\partial_0 = 0$.

To calculate $\Ima\partial_1$, let's compute:
$\partial_1(\alpha a + \beta b + \gamma c) = \alpha (v_1-v_0) + \beta (v_2-v_1) - \gamma (v_2-v_0) 
= (\gamma -\alpha)v_0 + (\alpha - \beta)v_1 + (-(\gamma - \alpha)-(\alpha - \beta))v_2 $ 
 \[ \Ima\partial_1 = \left\{ \left(\begin{array}{c}
          		                 	( \gamma - \alpha )\\
          		                 	(\alpha - \beta)\\
          		                 	-(\gamma - \alpha)-(\alpha - \beta)\\
          		                 \end{array} \right), \quad \alpha, \beta, \gamma \subseteq \mathbb{Z} \right\} 
          		                 \subseteq \mathbb{Z}^{\oplus^3} \]
    There exist an isomorphism $\Ima\partial_1 \simeq \mathbb{Z}^2$
    
$H_0 = \frac{\ker\partial_0}{\Ima\partial_1} = \mathbb{Z}^{3} \left/ {
    \left(\begin{array}{c}
                    1\\
                    0\\
                    -1\\
            \end{array} \right)\mathbb{Z} \oplus
    \left(\begin{array}{c}
                    0\\
                    1\\
                    -1\\
            \end{array} \right)\mathbb{Z}} \right. \simeq \mathbb{Z}$ 

    \end{frame}
 
  \begin{frame}
  \frametitle{$H_1$ - \# of holes}
        Second, let's compute $H_1$: \\
        $\ker\partial_1 = \left\{ \left(\begin{array}{c}
                            m\\
                            m\\
                            m\\
                    \end{array} \right), m \in \mathbb{Z} = \right\} = \left(\begin{array}{c}
                            1\\
                            1\\
                            1\\
                    \end{array} \right) \mathbb{Z} \simeq \mathbb{Z}$ \\ 
        $\Ima\partial_2 = \{0\}$ since $C_2 = \{0\}$ \\
        $H_1 = \frac{\ker\partial_1}{\Ima\partial_2} = 
                \frac{ \ker{\partial_1} }{ \{0\} } = \ker{\partial_1} \simeq \mathbb{Z}$ 

                
        Finally, the homology groups of the circle are: 
                \[
                    H_n^\Delta(S^1) \simeq \left\{
                        \begin{array}{rl}
                        \mathbb{Z}, & \textrm{for} \: n = 0, 1\\
                        
                                0 & \textrm{for} \: n \geqslant 2
                        \end{array}
                    \right.
                \]
 \end{frame}


  \section{Maps of Complexes and Maps on Homology}
  
  \begin{frame}
  \frametitle{Filtered Complex}
  \begin{figure}
                    $\mathcal{X}$  \hspace{3.5cm}  $\mathcal{Y}$  \hspace{3.5cm}  $\mathcal{Z}$ \\~\\
                    
                    
  \begin{tikzpicture}[line join = round, line cap = round]

                        % first-simplex
                        \coordinate [label=below:$v_0$] (0) at (0,0);
                        \coordinate [label=below:$v_1$] (1) at (2,0);
                        \coordinate [label=above:$v_2$] (2) at (1,{sqrt(3)});
                       
                         % second-simplex
                        \coordinate [label=below:$v_0$] (3) at (4,0);
                        \coordinate [label=below:$v_1$] (4) at (6,0);
                        \coordinate [label=above:$v_2$] (5) at (5,{sqrt(3)});
                        
                        % third-simpex
                        \coordinate [label=below:$v_0$] (6) at (8,0);
                        \coordinate [label=below:$v_1$] (7) at (10,0);
                        \coordinate [label=above:$v_2$] (8) at (9,{sqrt(3)});

                        \begin{scope}[decoration={markings,mark=at position 0.5 with
                            {\arrow{to}}}]
                            % first-simplex
                            \draw[fill] (0) circle [radius=0.03];
                            \draw[fill] (1) circle [radius=0.03];
                            \draw[fill] (2) circle [radius=0.03];
                            \draw [thin, right hook->] (2.75, 0.86) -- (3.25, 0.86);  
                            % second-simplex
                            \draw[fill] (3) circle [radius=0.03];
                            \draw[fill] (4) circle [radius=0.03];
                            \draw[fill] (5) circle [radius=0.03];
                            \draw[thick, postaction={decorate}] (3)--(4);
                            \draw[thick, postaction={decorate}] (3)--(5);
                            \draw[thick, postaction={decorate}] (4)--(5);
                           \draw [thin, right hook->] (6.75, 0.86) -- (7.25, 0.86);  
                           % third-simpex
                            \draw[fill] (6) circle [radius=0.03];
                            \draw[fill] (7) circle [radius=0.03];
                            \draw[fill] (8) circle [radius=0.03];
                            \draw[thick, postaction={decorate}] (6)--(7);
                            \draw[thick, postaction={decorate}] (6)--(8);
                            \draw[thick, postaction={decorate}] (7)--(8);
                            \filldraw[opacity=.3, gray] (6) --  (7) --  (8) -- cycle;
                        \end{scope}

                    \end{tikzpicture} 
                    
                    
                    
                     $C_\bullet^0  \longrightarrow   C_\bullet^1   \longrightarrow C_\bullet^2$  \\~\\
                    
                    
                    \end{figure}
 \end{frame}
 
 \begin{frame}
 \frametitle{Maps of Complexes}
 \footnotesize
 \setlength{\arraycolsep}{0.5pt}
 \medmuskip = 0mu % default: 4mu plus 2mu minus 4mu
 \[
                    \begin{tikzcd}[ampersand replacement=\&]
  \& C_\bullet^0   \arrow[rr]                                     \&  \& C_\bullet^1 \arrow[rr]                               \&  \& C_\bullet^2                                                                 \\ \\
  \& 0 \arrow[d]                                         \&  \& 0 \arrow[d]                                         \&  \& 0 \arrow[d]                                                                 \\ 
2 \& 0 \arrow[rr] \arrow[d, "\partial^0_2=0"]            \&  \& 0 \arrow[rr] \arrow[d, "\partial^1_2=0"]            \&  \& \mathbb{Z} \arrow[d, "\partial_2^2 = {\begin{pmatrix} 1 \\ 1  \\  -1  \end{pmatrix}}"] \\
1 \& 0 \arrow[d, "\partial^0_1=0"] \arrow[rr]            \&  \& \mathbb{Z}^3 \arrow[rr] \arrow[d, "\partial_1^1 = \partial_1^2"]                   \&  \& \mathbb{Z}^3 \arrow[d, "\partial_1^2 = {\begin{pmatrix} -1 & 0 & -1 \\ 1 & -1 & 0 \\ 0 & 1 & 1 \end{pmatrix}}"]                                                      \\
0 \& \mathbb{Z}^3 \arrow[d, "\partial^0_0=0"] \arrow[rr] \&  \& \mathbb{Z}^3 \arrow[rr] \arrow[d, "\partial^1_0=0"] \&  \& \mathbb{Z}^3 \arrow[d, "\partial^2_0=0"]                                    \\
  \& 0                                                   \&  \& 0                                                   \&  \& 0                                                                          
\end{tikzcd}
\]
    \end{frame}

    
 \begin{frame}
 \frametitle{Maps of Complexes induce maps on Homology}
 \footnotesize
 \setlength{\arraycolsep}{0.5pt}
 \medmuskip = 0mu % default: 4mu plus 2mu minus 4mu
 \[
                    \begin{tikzcd}[ampersand replacement=\&]
  \& H(C_\bullet^0)   \arrow[rr]                                     \&  \& H(C_\bullet^1) \arrow[rr]                               \&  \& H(C_\bullet^2)                                                                 \\ \\
  \& 0 \arrow[d]                                         \&  \& 0 \arrow[d]                                         \&  \& 0 \arrow[d]                                                                 \\ 
2 \& 0 \arrow[rr] \arrow[d]            \&  \& 0 \arrow[rr] \arrow[d]            \&  \& 0 \arrow[d] \\
1 \& 0 \arrow[d] \arrow[rr]            \&  \& \mathbb{Z} \arrow[rr] \arrow[d]                   \&  \& 0 \arrow[d]                                              \\
0 \& \mathbb{Z}^3 \arrow[d] \arrow[rr] \&  \& \mathbb{Z} \arrow[rr] \arrow[d] \&  \& \mathbb{Z} \arrow[d]                                    \\
  \& 0                                                   \&  \& 0                                                   \&  \& 0                                                                          
\end{tikzcd}
\]
    \end{frame}
    
    
    \section{\v{C}ech and Vietoris-Rips Complex}
    \subsection{Definition}
 \begin{frame}
If $\mathcal{X}$ is a metric space and $r \geq 0$:
\begin{definition}
 The \v{C}ech Complex has vertix set $\mathcal{X}$ and simplex $[v_0, v_1, ..., v_n]$ when
 \[
  \bigcap_{i = 0}^n \mathcal{B}(v_i; r/2) \neq \varnothing
 \]

 \end{definition}
 \begin{definition}
 The Vietoris Rips Complex has vertix set $\mathcal{X}$ and simplex $[v_0, v_1, ..., v_n]$ when
 \[
  d(v_i, v_j) \leq r \; \forall i, j
  \]

 \end{definition}
 
 Relation of \v{C}ech and Vietoris-Rips Complex: 
 For each $\epsilon > 0$, there is a chain inclusion maps
    \[
     \mathcal{R} \hookrightarrow \mathcal{C}_{\epsilon\sqrt{2}} \hookrightarrow \mathcal{R}_{\epsilon\sqrt{2}}
    \]

 
 \end{frame}


  \subsection{\v{C}ech Complex Example}
  \begin{frame}
  \frametitle{\v{C}ech Complex Example}
  \begin{tikzpicture}[line join = round, line cap = round]

                        % first-simplex
                        \coordinate (0) at (0,0);
                        \coordinate (1) at (2,0);
                        \coordinate (2) at (1,{sqrt(3)});
                       
                         % second-simplex
                        \coordinate  (3) at (4,0);
                        \coordinate  (4) at (6,0);
                        \coordinate  (5) at (5,{sqrt(3)});
                        
                        % third-simpex
                        \coordinate  (6) at (8.5,0);
                        \coordinate  (7) at (10.5,0);
                        \coordinate  (8) at (9.5,{sqrt(3)});

                        \begin{scope}[decoration={markings,mark=at position 0.5 with
                            {\arrow{to}}}]
                            
                            % first-simplex
                            \draw[fill] (0) circle [radius=0.05];
                            \draw[fill, opacity=0.3, blue] (0) circle [radius=0.3];
                            \draw[fill] (1) circle [radius=0.05];
                            \draw[fill, opacity=0.3, blue] (1) circle [radius=0.3];
                            \draw[fill] (2) circle [radius=0.05];
                            \draw[fill, opacity=0.3, blue] (2) circle [radius=0.3];
                            \draw [thin, right hook->] (2.75, 0.86) -- (3.25, 0.86);  
                            
                            % second-simplex
                            \draw[fill] (3) circle [radius=0.05];
                            \draw[fill, opacity=0.3, blue] (3) circle [radius=1.05];
                            \draw[fill] (4) circle [radius=0.05];
                            \draw[fill, opacity=0.3, blue] (4) circle [radius=1.05];
                            \draw[fill] (5) circle [radius=0.05];
                            \draw[fill, opacity=0.3, blue] (5) circle [radius=1.05];
                            \draw[thick] (3)--(4);
                            \draw[thick] (3)--(5);
                            \draw[thick] (4)--(5);
                            \draw [thin, right hook->] (6.75, 0.86) -- (7.25, 0.86);  
                            
                           % third-simpex
                            \draw[fill] (6) circle [radius=0.05];
                            \draw[fill, opacity=0.3, blue] (6) circle [radius=1.2];
                            \draw[fill] (7) circle [radius=0.05];
                            \draw[fill, opacity=0.3, blue] (7) circle [radius=1.2];
                            \draw[fill] (8) circle [radius=0.05];
                            \draw[fill, opacity=0.3, blue] (8) circle [radius=1.2];
                            \draw[thick] (6)--(7);
                            \draw[thick] (6)--(8);
                            \draw[thick] (7)--(8);
                            \filldraw[opacity=.5, gray] (6) --  (7) --  (8) -- cycle;
                        \end{scope}

                    \end{tikzpicture} 
                    
 \end{frame}
 
 \subsection{Vietoris-Rips Complex Example}
 \begin{frame}
  \frametitle{Vietoris-Rips Complex Example}
  \begin{tikzpicture}[line join = round, line cap = round]

                       % first-simplex
                        \coordinate (0) at (0,0);
                        \coordinate (1) at (2,0);
                        \coordinate (2) at (1,{sqrt(3)});
                       
                         % second-simplex
                        \coordinate  (3) at (4,0);
                        \coordinate  (4) at (6,0);
                        \coordinate  (5) at (5,{sqrt(3)});
                        
                        % third-simpex
                        \coordinate  (6) at (8.5,0);
                        \coordinate  (7) at (10.5,0);
                        \coordinate  (8) at (9.5,{sqrt(3)});

                        \begin{scope}[decoration={markings,mark=at position 0.5 with
                            {\arrow{to}}}]
                            % first-simplex
                            \draw[fill] (0) circle [radius=0.05];
                            \draw[fill, opacity=0.3, blue] (0) circle [radius=0.3];
                            \draw[fill] (1) circle [radius=0.05];
                            \draw[fill, opacity=0.3, blue] (1) circle [radius=0.3];
                            \draw[fill] (2) circle [radius=0.05];
                            \draw[fill, opacity=0.3, blue] (2) circle [radius=0.3];
                            \draw [thin, right hook->] (2.75, 0.86) -- (3.25, 0.86);  
                            % second-simplex
                            \draw[fill] (3) circle [radius=0.05];
                            \draw[fill, opacity=0.3, blue] (3) circle [radius=1.05];
                            \draw[fill] (4) circle [radius=0.05];
                            \draw[fill, opacity=0.3, blue] (4) circle [radius=1.05];
                            \draw[fill] (5) circle [radius=0.05];
                            \draw[fill, opacity=0.3, blue] (5) circle [radius=1.05];

                            \draw[thick] (3)--(4);
                            \draw[thick] (3)--(5);
                            \draw[thick] (4)--(5);
                           \draw [thin, right hook->] (6.75, 0.86) -- (7.25, 0.86);  
                            \filldraw[opacity=.5, gray] (3) --  (4) --  (5) -- cycle;
                           % third-simpex
                            \draw[fill] (6) circle [radius=0.05];
                            \draw[fill, opacity=0.3, blue] (6) circle [radius=1.2];
                            \draw[fill] (7) circle [radius=0.05];
                            \draw[fill, opacity=0.3, blue] (7) circle [radius=1.2];
                            \draw[fill] (8) circle [radius=0.05];
                            \draw[fill, opacity=0.3, blue] (8) circle [radius=1.2];
                            \draw[thick] (6)--(7);
                            \draw[thick] (6)--(8);
                            \draw[thick] (7)--(8);
                            \filldraw[opacity=.5, gray] (6) --  (7) --  (8) -- cycle;
                        \end{scope}

                    \end{tikzpicture} 
                    
 \end{frame}
 \begin{frame}
\begin{figure}
    \includegraphics[width=80mm,scale=1.5]{rips_of_annulus.png}
    \caption{A sequence of Rips Complex from a point clound data set that represent an annulus}
    \label{fig:fig1}
    \end{figure}

            \end{frame}
    
\section{Persistent Homology}
  \subsection{Persistance}
 \begin{frame}
 \begin{definition}
 
    Given a filtered complex, the i-th complex $K^i$ has associated boundary operators 
    $\partial^i_k_$, matrices $M^i_k$ , and groups $C^i_k$, $Z^i_k$, $B^i_k$, and $H^i_k$ for all $i, k \geq 0$
    The p-persistent k-th homology group of $K^i$ is 
            \[
             H_k^{i, p} = Z_k^i \left/ (B_k^{i+p} \cap Z_k^i) \right.
            \]

            \end{definition}
             \vspace{0.5in}
             Example: 
             p = i = k = 1: $ H_1^{1, 1} = Z_1^1 \left/ (B_2^{1} \cap Z_1^1) \right. \mathbb{Z} \left/ ( \{0\} \cap \mathbb{Z}) \right. = \mathbb{Z} $ \\~\\

            \end{frame}
 
\begin{frame}
\begin{figure}
    \includegraphics[width=80mm,scale=1.5]{barcodes.png}
    \caption{An example of barcode representations of the homology of the sampling of
points in an annulus }
    \label{fig:fig1}
    \end{figure}

    \end{frame}

  
 \subsection{Computations}
 \begin{frame}
 \begin{figure}
  \includegraphics[width=80mm,scale=1.5]{circle.png}
    \caption{10 points on a cicle}
    \label{fig:fig1}
    \end{figure}
 \end{frame}
  
  \begin{frame}
 \begin{figure}
  \includegraphics[width=80mm,scale=1.5]{Circle_homology.png}
    \caption{Persistent diagram of homology of circle (10 points)}
    \label{fig:fig1}
    \end{figure}
 \end{frame}
 \begin{frame}
 \begin{figure}
  \includegraphics[width=80mm,scale=1.5]{circle_lifetime.png}
    \caption{Lifetime diagram of homology of circle (10 points)}
    \label{fig:fig1}
    \end{figure}
 \end{frame}
 \begin{frame}
 \begin{figure}
  \includegraphics[width=80mm,scale=1.5]{timeseries.png}
    \caption{Peridicity in Timeseries}
    \label{fig:fig1}
    \end{figure}
 \end{frame}
  \begin{frame}
  \frametitle{References}
  \begin{itemize}
   \item A. Zomorodian and G. Carlsson, �Computing Persistent Homology,� Discrete Comput.
Geom., 33, (2005), 249�274.
   \item  A. Hatcher, Algebraic Topology, Cambridge University Press, (2002).
   \item  V. de Silva and G. Carlsson. �Topological estimation using witness complexes,� in SPBG04
Symposium on Point-Based Graphics (2004), 157-166
    \item Perea, J.A., Harer, J. Sliding Windows and Persistence: An Application of Topological Methods to Signal Analysis. Found Comput Math 15, 799�838 (2015).
    \item H. Kantz and T. Schreiber, Nonlinear Time Series Analysis, Cambridge University Press, 2003.
    \tiem Ghrist, Robert. (2008). Barcodes: The persistent topology of data. BULLETIN (New Series) OF THE AMERICAN MATHEMATICAL SOCIETY. 45. 10.1090/S0273-0979-07-01191-3. 
  \end{itemize}

\end{frame}
 

 
\end{document}
